The first finite difference scheme considered, was the one used by Holden and Raynaud \cite{holden2006convergence}:
\begin{align}
m &= u - D_{-}D_{+}u, \notag \\ 
m_t &= -D_{-}(mu) - mD(u),
\end{align}
where the operators are defined as
\begin{align}
\label{eq:operators}
D_{+}u_{j}^{n} = \frac{u_{j}^{n}-u_{j-1}^{n}}{h},\quad
D_{-}u_{j}^{n} = \frac{u_{j+1}^{n}-u_{j}^{n}}{h},\quad
D = \frac{D_{+}+D_{-}}{2}
\end{align}
The main disadvantage of this scheme is that one has to assume only positive values of $u$.

Our finite difference scheme is based on a scheme presented by Morten Lien Dahlby\cite{dahlby2007geometric}, and is a slight modification of the scheme presented by Holden and Raynaud.

\begin{align}
m &= u - D_{-}D_{+}u, \notag \\ 
m_t &= -D_{-}(m(u \vee 0)) -D_{+}(m(u \wedge 0)) - mD(u), 
\end{align}
where $(u \vee 0) = \text{max}(u,0)$ and $(u \wedge 0) = \text{min}(u,0)$.

The original scheme by Holden and Raynaud used the $D_{-}$-operator, making it an upwind method. This would not work for antipeakons, peakons with negative height and speed. Modifying the scheme to use  $D_{-}$ when $u > 0$ and $D_{+}$ when $ u < 0$ makes it applicable to waves traveling in either direction. 

To find an appropriate temporal step, the Courant–Friedrichs–Lewy (CFL) condition was considered. The CFL condition is a necessary condition for stability in our finite difference scheme, and can be stated as
\begin{align}
c\frac{\Delta t}{\Delta x} \leq 1,
\end{align}
where $c$ is the velocity of the wave, $\Delta t$ is the length of the temporal step, and $\Delta x$ is the length of the spacial step. Rearranging this inequality gives an expression for $\Delta t$:
\begin{align}
\Delta t \leq \frac{\Delta x}{c}
\end{align}
This makes it necessary to find an estimate for $c$. This was done by finding the integral of the initial condition, followed by calculating the height of a peakon of the form $c\text{e}^{-|x-ct|}$ with the same area, as shown in Figures \ref{fig:area1} and \ref{fig:area2}. Using the fact that a peakon's height is proportional to its speed, we have an approximation of $c$, and therefore also an approximation of $\Delta t$. The calculation of $\Delta t$ for antipeakons is analogous. \\
\begin{figure}
\begin{subfigure}[b]{0.49\textwidth}
                \includegraphics[width=\textwidth]{gfx/areainitial}
                \caption{Initial condition}
                \label{fig:area1}
\end{subfigure}
\begin{subfigure}[b]{0.49\textwidth}
                \includegraphics[width=\textwidth]{gfx/areapeakon}
                \caption{Peakon of same area}
                \label{fig:area2}
\end{subfigure}
\caption{Initial condition and peakon of same area, used to estimate $c$.}
\end{figure}


A preliminary conclusion based on the plot of errors for decreasing spacial stepsize $h$ (Figure \ref{fig:erroroftime}) shows that decreasing $h$ will improve the approximation. Figure \ref{fig:attimeT} shows that as the spacial stepsize $h$ decreases, our scheme becomes a good approximation of the solution. \\


Our numerical experiments show a rate of convergence in space which is slightly faster than a linear rate (Figure \ref{fig:loglog}). This was calculated using a short time interval, to reduce the effects of propagation of error. However, we experience that decreasing the temporal stepsize beyond a certain size increases the error. 

\begin{figure}[h]
        \centering
        \includegraphics[width=0.8\textwidth]{gfx/erroroftime}
        \caption{Error plots for decreasing $h$, constant time and space interval.}
        \label{fig:erroroftime}
\end{figure}

\begin{figure}[h]
        \centering
        \includegraphics[width=0.8\textwidth]{gfx/attimeT}
        \caption{Plot of approximated solution together with analytical solution for decreasing spacial step $h$.}
        \label{fig:attimeT}
\end{figure}

\begin{figure}[h]
        \centering
        \includegraphics[width=0.8\textwidth]{gfx/loglog}
        \caption{Loglog plot - rate of convergence in space. In this plot, the red reference line has a slope of 1, while the blue line found through numerical experiments has a slope of 1.08}
        \label{fig:loglog}
\end{figure}
