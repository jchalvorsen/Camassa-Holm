\subsection*{Transforming $m$ to $u$}

For each time step when solving the scheme shown in equation (\ref{eq:modscheme}), one essentially transforms $u$ to $m$ by the transformation matrix $\bm{A}$, such that $\bm{m} = \bm{Au}$. Transforming $m$ back to $u$ usually involves solving the linear system $\bm{Au} = \bm{m}$. 

Holden and Raynaud \cite{holden2006convergence} present a more efficient way of transforming $m$ to $u$. Denoting the linear operator from $u$ to $m$ as $L$, $u$ can be written as a discrete convolution product of $m$ and $g$:
\begin{align*}
u_i = L^{-1} m_i = \sum\limits_{j \in \mathds{Z}} g_{i - j} m_j,
\end{align*}

where $g$ is periodized such that
\begin{align*}
g_i^p = c \frac{e^{-\kappa i} + e^{\kappa (i - n)}}{1 - e^{-\kappa n}},
\end{align*}

where $i = 0, ..., n - 1$, and $g$ is defined by the constants
\begin{equation}
\label{eq:g_constants}
\begin{aligned}
\kappa = \ln \left( 1 + \frac{h(\sqrt{4 + h^2} + h)}{2} \right)
\end{aligned}
\qquad
\begin{aligned}
c = \frac{1}{1 + \frac{2}{h^2} (1 - e^{-\kappa})}.
\end{aligned}
\end{equation}

Note that when comparing to the original paper, the constants here are defined in a slightly different way. The reason is simply that Holden and Raynaud derive $g$ on the interval $x \in [0, 1]$, whereas \eqref{eq:g_constants} represents the equivalent extension to any interval.

The convolution can then efficiently be calculated by the Fast Fourier Transform (FFT):
\begin{align*}
u = \mathcal{F}^{-1} (\mathcal{F} (g) \cdot \mathcal{F}(m)),
\end{align*}

where $\mathcal{F}$ denotes the FFT and $\mathcal{F}^{-1}$ denotes the inverse FFT. Since $g$ is constant, its fourier transform can be precalculated, which saves a lot of computation time. In fact, compared to solving the linear system by standard matrix techniques, we measured the scheme to consistently spend a staggering $77\%$ less time when implemented with the FFT method.

\subsection*{Computation time, memory usage and high-resolution grids}
Through active use of MATLAB's profiling tools, we have eliminated any substantial bottlenecks that are not directly required for the scheme to work. One could however shave another estimated $5-10\%$ off the running time at the cost of code clarity and readability, but at that point you may want to consider implementations in a more efficient language. In fact, the scheme is so efficient that computational time is typically not an issue unless one is generating a grid whose size is approaching or beyond the amount of memory found on a regular computer.

Due to the slow convergence of the scheme, one may want to generate very large grids to achieve high-resolution simulations for extended time periods. In such cases, the memory required to hold the grid may exceed any practical memory capacity available. As a workaround, we tested a technique where we ran the simulation in a stepwise manner, such that the interval $[0, T]$ would be subdivided into a number of smaller intervals $[\tau_j, \tau_{j+1}]$ for $j = 0, ..., J - 1$, where $J$ is the number of subdivisions. Whenever a subdivision completes, the resulting matrix of that computation is compressed into a smaller matrix by interpolation of the grid, and the next subdivision is computed, using the results at time $\tau_{j}$ from the previous subdivision as its initial condition. Concatenating the compressed matrices yields a reasonably sized matrix whose structure is an accurate approximation of the high-resolution structure. Since the interpolation introduces (typically very small) additional errors, it is not a suitable technique for convergence verification, but it is very applicable in real world scenarios.