A linearization of our finite difference scheme was performed, in an attempt to say something about the nonlinear scheme. 

\begin{align}
m &= u - D_{-} D_{+} u \\
m_{t} &= -uD_{-}m - \frac{3}{2}mD_{-}u - \frac{1}{2}mD_{+}u
\end{align} 

Define $f(U) := f(u, D_{-}u, D_{+}u, m, D_{-}m),$ so that the Taylor expansion becomes
\begin{align}
f(U) &\approx f(U_{0}) + f'(U_{0})^{T}(U-U_{0}),\\
&= f(U_{0}) - f'(U_{0})^{T}U_{0} + f'(U_{0})^{T}U, \\
\end{align}
where the first two terms are known, depending only on a chosen function $U_{0}$. Expanding the terms eventually leads to the linear finite difference scheme
\begin{align}
m_{t} = f(U) \approx \alpha + A_{0}u + B_{0}D_{+}u + C_{0}D_{-}u + D_{0}m + E_{0}D_{-}m,
\end{align}
with terms
\begin{align}
A_{0} &= -D_{-}m_{0}, \\
B_{0} &= - \frac{1}{2}m_{0}, \\
C_{0} &= - \frac{3}{2}m_{0}, \\
D_{0} &= - \frac{3}{2}D_{-}u_{0} - \frac{1}{2}D_{+}u_{0}, \\
E_{0} &= - u_{0}, \\
\alpha &= 2 D_0 B_0 + A_0 E_0
\end{align}

Unfortunately the linearized scheme proved to be unstable. In an effort to stabilize it, numerical experimentation showed that changing a few of the terms yielded a scheme that when linearized about a peakon solution yielded a stable and reasonably accurate approximation in the single peakon case. The new terms are

\begin{align}
m &= -(u - D_{-}D_{+}u), \\
D_{0} &= -\frac{3}{2}(D_{-}u + D_{+}u), \\
\alpha &= -2D_{0}B_{0} + A_{0}E_{0}.
\end{align}